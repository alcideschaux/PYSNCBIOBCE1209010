% Options for packages loaded elsewhere
\PassOptionsToPackage{unicode}{hyperref}
\PassOptionsToPackage{hyphens}{url}
%
\documentclass[
]{book}
\usepackage{lmodern}
\usepackage{amssymb,amsmath}
\usepackage{ifxetex,ifluatex}
\ifnum 0\ifxetex 1\fi\ifluatex 1\fi=0 % if pdftex
  \usepackage[T1]{fontenc}
  \usepackage[utf8]{inputenc}
  \usepackage{textcomp} % provide euro and other symbols
\else % if luatex or xetex
  \usepackage{unicode-math}
  \defaultfontfeatures{Scale=MatchLowercase}
  \defaultfontfeatures[\rmfamily]{Ligatures=TeX,Scale=1}
\fi
% Use upquote if available, for straight quotes in verbatim environments
\IfFileExists{upquote.sty}{\usepackage{upquote}}{}
\IfFileExists{microtype.sty}{% use microtype if available
  \usepackage[]{microtype}
  \UseMicrotypeSet[protrusion]{basicmath} % disable protrusion for tt fonts
}{}
\makeatletter
\@ifundefined{KOMAClassName}{% if non-KOMA class
  \IfFileExists{parskip.sty}{%
    \usepackage{parskip}
  }{% else
    \setlength{\parindent}{0pt}
    \setlength{\parskip}{6pt plus 2pt minus 1pt}}
}{% if KOMA class
  \KOMAoptions{parskip=half}}
\makeatother
\usepackage{xcolor}
\IfFileExists{xurl.sty}{\usepackage{xurl}}{} % add URL line breaks if available
\IfFileExists{bookmark.sty}{\usepackage{bookmark}}{\usepackage{hyperref}}
\hypersetup{
  pdftitle={Formación en bioética de docentes y alumnos de medicina y ciencias de la salud},
  pdfauthor={Alcides Chaux, M.D.},
  hidelinks,
  pdfcreator={LaTeX via pandoc}}
\urlstyle{same} % disable monospaced font for URLs
\usepackage{longtable,booktabs}
% Correct order of tables after \paragraph or \subparagraph
\usepackage{etoolbox}
\makeatletter
\patchcmd\longtable{\par}{\if@noskipsec\mbox{}\fi\par}{}{}
\makeatother
% Allow footnotes in longtable head/foot
\IfFileExists{footnotehyper.sty}{\usepackage{footnotehyper}}{\usepackage{footnote}}
\makesavenoteenv{longtable}
\usepackage{graphicx,grffile}
\makeatletter
\def\maxwidth{\ifdim\Gin@nat@width>\linewidth\linewidth\else\Gin@nat@width\fi}
\def\maxheight{\ifdim\Gin@nat@height>\textheight\textheight\else\Gin@nat@height\fi}
\makeatother
% Scale images if necessary, so that they will not overflow the page
% margins by default, and it is still possible to overwrite the defaults
% using explicit options in \includegraphics[width, height, ...]{}
\setkeys{Gin}{width=\maxwidth,height=\maxheight,keepaspectratio}
% Set default figure placement to htbp
\makeatletter
\def\fps@figure{htbp}
\makeatother
\setlength{\emergencystretch}{3em} % prevent overfull lines
\providecommand{\tightlist}{%
  \setlength{\itemsep}{0pt}\setlength{\parskip}{0pt}}
\setcounter{secnumdepth}{5}
\usepackage{booktabs}
\usepackage{amsthm}
\makeatletter
\def\thm@space@setup{%
  \thm@preskip=8pt plus 2pt minus 4pt
  \thm@postskip=\thm@preskip
}
\makeatother
\usepackage[]{natbib}
\bibliographystyle{apalike}

\title{Formación en bioética de docentes y alumnos de medicina y ciencias de la salud}
\author{Alcides Chaux, M.D.}
\date{Fecha de última actualización: 2021-01-02}

\begin{document}
\maketitle

{
\setcounter{tocdepth}{1}
\tableofcontents
}
\hypertarget{propuxf3sito}{%
\chapter{Propósito}\label{propuxf3sito}}

Evaluamos el nivel de formación en bioética de docentes y alumnos de las áreas de medicina y ciencias de la salud. Anali-zamos 105 docentes y 7 programas académicos (Bioquímica, Enfermería, Fisioterapia y Kinesiología, Medicina, Nutrición, Odontología y Psicología). Encontramos un bajo porcentaje de entrenamiento formal en docentes y una insuficiente formación ofrecida a los alumnos en el área de bioética.

\hypertarget{resumen}{%
\chapter{Resumen}\label{resumen}}

La bioética es una disciplina filosófica que trata acerca de las cuestiones sociales, legales, culturales, epidemiológicas, ecológicas y éticas que surgen en el campo de las ciencias de la salud y de la vida. La formación en bioética es parte esencial de la educación universitaria, particularmente en las carreras del área de medicina y ciencias de la salud. La enseñanza de la bioética debería brindar las bases para que los profesionales del área de la salud (ya sea en formación o en ejercicio) desempeñen sus actividades con valores éticos firmes en un ambiente cambiante cada vez más complejo desde el punto de vista social, cultural y ecológico. Por lo tanto, es fundamental una sólida enseñanza en el área de bioética, la que debe desarrollarse tanto formalmente en la malla curricular como transversalmente mediante la aplicación de los principios bioéticos fundamentales en los distintos cursos que componen el programa de estudios de las carreras del área de salud. Mediante este estudio, evaluamos el nivel de formación en bioética de alumnos y docentes de las carreras de medicina y ciencias de la salud de la Universidad del Norte (Asunción, Paraguay), con el propósito de establecer una línea de base diagnóstica sobre la cual desarrollar estrategias para mejorar dicha formación. Con este fin, analizamos los registros académicos de los docentes de las carreras de las áreas evaluadas (Medicina, Enfermería, Nutrición, Bioquímica, Odontolo-gía, Psicología, y Fisioterapia y Kinesiología), así como los programas de estudio que forman parte del plan curricular 2020 de dichas carreras. Se obtuvieron re-gistros académicos completos de 105 docentes. Observamos que sólo un cuarto de estos docentes recibió algún tipo de entrenamiento en bioética, principalmen-te entrenamientos no formales tales como cursos de corta duración. El porcenta-je de docentes que recibió entrenamiento formal (a nivel de especialización o maestría) fue más bajo aún (2\%). Con respecto al nivel de formación ofrecido a los alumnos, de las 7 carreras evaluadas, notamos que en todas ellas se incluyó sólo una asignatura específica del área de ética, siendo esta asignatura semes-tral, con una carga horaria de 2 a 5 horas semanales. También notamos la in-clusión esporádica de temas bioéticos en los programas de estudio de otras asig-naturas. El énfasis estuvo puesto en ética profesional y deontología, con escasa o nula inclusión de otros temas pertinentes al área de bioética. Más aún, la me-todología de enseñanza fue predominantemente mediante clases magistrales, no evidenciándose otras estrategias tales como la discusión de casos para el desa-rrollo del pensamiento crítico por parte de los alumnos o el aprendizaje basado en la resolución de problemas. Nuestros hallazgos indican una deficiente forma-ción bioética, tanto en docentes como en alumnos de las áreas de medicina y ciencias de la salud. Esto resalta la necesidad de implementar estrategias de me-joramiento en el área de bioética, las que redundarán en beneficio de los profe-sionales en formación tanto en sus vidas personales, su entorno social, como en el ámbito clínico del ejercicio de su profesión.

\hypertarget{palabras-clave-mesh}{%
\section{Palabras clave (MeSH)}\label{palabras-clave-mesh}}

bioética (D001675), ética clínica (D026690), entrena-miento docente (D000070260), alumnos (D013334), currículo (D003479)

  \bibliography{book.bib,packages.bib}

\end{document}
